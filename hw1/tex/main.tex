\documentclass{article}
\linespread{1.075}
\usepackage{times}

\usepackage{physics}

\usepackage{listings}
\lstset{
    basicstyle=\ttfamily\footnotesize,
    numbers=left,
    frame=single,
}

\usepackage{hyperref}
\hypersetup{
    colorlinks=true,
    urlcolor=red,
}

\usepackage{graphicx}
\graphicspath{ {../gnuplot/plots} }

\begin{document}

{\centering

Ph 121c

Assignment 1

Lorenzo Van Munoz

\today

}

\newpage


All of my code follows:

\newpage

{\bf \noindent
Contents of dense\_8\_10\_12.f90 and dense\_14.f90
}

Note: I split $L \in \{8, 10, 12\}$ and $L=14$ otherwise
I had segfault/malloc errors.

\lstinputlisting[language=Fortran]{../fortran/dense_8_10_12.f90}

\lstinputlisting[language=Fortran]{../fortran/dense_14.f90}

\newpage

{\bf \noindent
Contents of tfim.f90
}

\lstinputlisting[language=Fortran]{../fortran/tfim.f90}

\newpage

{\bf \noindent
Contents of lanczos.f90
}

\lstinputlisting[language=Fortran]{../fortran/lanczos.f90}

\newpage

{\bf \noindent
Computing environment
}

\lstinputlisting{include/versions.txt}


{\bf \noindent
Fortran compiler invocations
}

\begin{lstlisting}
$ cd hw1/fortran
$ ifx -c tfim.f90 lanczos.f90
$ ifx -qmkl tfim.o dense_8_10_12.f90 -o bin/dense_8_10_12.out -lmkl_lapack95_lp64
$ ifx -qmkl tfim.o dense_14.f90 -o bin/dense_14.out -lmkl_lapack95_lp64
\end{lstlisting}

\end{document}

