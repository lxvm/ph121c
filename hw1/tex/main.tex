\documentclass{article}
\linespread{1.075}
\usepackage{times}

\usepackage{physics}

\usepackage{listings}
\lstset{
    basicstyle=\ttfamily\footnotesize,
    numbers=left,
    frame=single,
}

\usepackage{hyperref}
\hypersetup{
    colorlinks=true,
    urlcolor=red,
}

\usepackage{graphicx}
\graphicspath{ {../gnuplot/plots} }

\begin{document}

{\centering

Ph 121c

Assignment 1

Lorenzo Van Munoz

\today

}

\tableofcontents

\newpage

\section{
Dense ED
}



\newpage

\section{
Sparse ED
}



\newpage

\section{
Convergence with system size
}



\newpage

\section{
Finding the quantum phase transition
}



\newpage

\section{
Magnetic ordering
}



\newpage

\section{
Using Ising symmetry
}



\newpage

\section{
Appendix: Code
}

All of my source code is available by viewing or cloning
\href{https://github.com/lxvm/ph121c.git}{this git repository}.

I had to duplicate some of my programs in two parts with
different parameters but otherwise identical code since 
trying the whole parameter sweep in one go gave rise to
segmentation fault errors inside of external routines.
This is yet another example of the Heisenbug uncertainty principle.

The remaining sections are a minimal reference for how the code
should be run and with what tools

\subsection{
Computing environment
}

I am using the Intel oneAPI base toolkit (with MKL)
and HPC toolkit (with Fortran), which is freely available
\href{https://software.intel.com/content/www/us/en/develop/
articles/free-intel-software-developer-tools.html}{here}.
This is some information about my system and software:

\lstinputlisting{./include/versions.txt}

This pdf document was generated by latexmk using \TeX\ Live 2020.

\subsection{
Fortran compiler invocations
}

These lines should be executed within the {\tt hw1/fortran }
directory, after executing the oneAPI {\tt setvars.sh } script.

\lstinputlisting{./include/compile_lines.txt}

The compiled binaries will be in the {\tt hw1/fortran/bin }
directory, but you must run them from the {\tt hw1/fortran } directory.
Their outputs will be saved in the {\tt hw1/fortran/data } directory.

\subsection{
Gnuplot plotting invocation
}

These lines should be executed within the {\tt hw1/gnuplot }
directory.

\lstinputlisting{include/make_plots.txt}

\end{document}
