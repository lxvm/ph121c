\documentclass{article}
\linespread{1.075}
\usepackage{times}

\usepackage{listings}
\lstset{
    basicstyle=\ttfamily\footnotesize,
    numbers=left,
    frame=single,
}

\usepackage{hyperref}
\hypersetup{
    colorlinks=true,
    urlcolor=red,
}

\usepackage{graphic}
\graphicspath{ {./gnuplot/plots} }

\begin{document}

{\centering

Ph 121c

Assignment 1

Lorenzo Van Munoz

\today

}

\newpage

I want to start by explaining the work I've done to setup my computer for
the computational tasks we will explore in this course.
I've spent a fair portion of the weekend setting up the computing resources I
will need and thought I should at least provide and overview of my computing
tools and environment.
\par

My personal computer is a Dell laptop, model xps 15 9570, which is currently
in a dual-boot configuration of Windows 10 (education) and Debian 11 (bullseye).
Debian is one of the most popular, free, and well-supported Linux
distributions, so I decided I will use it to do my work for this class because
of the prevalence of Linux in scientific computing
(for example, at \href{https://www.nas.nasa.gov/hecc/}{NASA}).
\par

The next major components are my computing tools.
Since my laptop has an intel core i7 processor, I decided by best options for
performance would be to use the proprietary compilers for the chip.
(Intel freely distributes its oneAPI toolkit
\href{https://software.intel.com/content/www/us/en/develop/articles/free-intel-software-developer-tools.html}{here}.)
I installed the base toolkit and HPC toolkit with an interest in using libmkl,
intel-python, ifort, and intel's profilers.
\par

A very high level summary of my computer environment and dependencies used for
the computation and visualization tasks in the assignment is the following:

\lstinputlisting{versions.log}

\end{document}

